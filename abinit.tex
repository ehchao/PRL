\section{Abinit}
\label{sec-abinit}
\textit{Abinit} est un logiciel qui implémente numériquement la théorie de la fonctionnelle de la densité.
Il nous permet de faire des calculs ab initio pour obtenir la densité électronique,
et ainsi l'énergie fondamentale du système.
Plus précisément, Abinit effectue la résolution auto-consistente pour le système d'équations de Kohn et Sham à l'aide une base d'ondes planes.
A chaque itération, le point minimum d'énergie est calculé par la méthode du gradient conjugué,
et l'approximation du pseudo-potentiel est introduite pour éliminer dans le systéme d'équations les électrons du cœur,
ce qui réduit considérablement la complexité du probleme.

Avant de démarer le calcul,
plusieurs paramètres doivent être spécifiés pour la résolution numérique.
Du côté physique, le système étudié étant cristallin,
il faut d'abord définir la maille élémentaire.
Du côté technique, il faut préciser les méthodes utilisées et des valuers critiques qui gouvernent le comportement du calcul.
Dans la suite, nous proposons d'abord quelques approximations importantes. Ensuite,
Nous présentons les paramètres principaux qui nous servent dans notre études.
Une liste complète des paramètres utilisés se trouve dans l'annexe.

\subsection{Espace réciproque}
\label{subsec-reciprocal}
La partie éssentielle du calcul avec Abinit consiste à résoudre le système d'équations de Kohn et Sham,
dont une quantité clée est la densité de charge $n$, qui est donnée par l'équation %TODO
Or les système étudiés sont souvent des matériaux cristallins,
auxquels on applique la condition périodique aux limites.
D'après le théorème de Bloch, les états propres, et ainsi les fonctions d'ondes correspondantes,
sont caractérisés par deux nombres quantiques:
le vecteur d'onde $k$ dans la première zone de Brillouin, et l'indice de bande $n$.
La densité de charge s'exprime donc par
\begin{equation}
  \label{eqn-den1}
  n(r) = \frac{\Omega_{maille}}{{(2\pi)}^3} \sum_{n, k}^{occ}|\phi_{n, k}(r)|^2,
\end{equation}
avec la somme sur tous les états occupés.
A la limite des cristaux idéals infinis, la taille de grille dans l'espace réciproque tend vers zéro,
alors~\cref{eqn-den1} devient
\begin{equation}
  \label{eqn-den2}
  n(r) = \frac{\Omega_{maille}}{{(2\pi)}^3} \sum_n^{occ}\int_{ZB}d^3k|\phi_{n, k}(r)|^2.
\end{equation}
Ensuite, à l'approximation numérique,
l'intégral sur $k$ doit être discrétisée en somme sur $N_k$ point $k$ pondérés:
\begin{equation}
  \label{eqn-den3}
  n(r) = \sum_n^{occ}\sum_{i=1}^{N_k}w_i|\phi_{n, k}(r)|^2.
\end{equation}
Le temps de calcul étant proportionnel à $N_k$,
il est donc nécessaire de choisir les poids $w_i$ et les points $k$ afin de reproduire le résultat
de l'intégral avec la moindre erreur et le moindre nombre de point $k$.
Une méthode beaucoup utilisée est l'échantillonage de Monkhorst-Pack,
qui consiste en pricipal à prendre les points $k$ tels que chaque point ne peut pas être lié avec un autre par aucun opération de symétrie,%TODO:not clear enough
avec un certain décalage si nécessaire.
La justification de la méthode est donnée dans.%TODO:find reference
Avec cette méthode, le nombre de points nécessaire pour évaluer l'intégral dans la première zone de Brillouin est reduit énoremément.

\subsection{Base d'ondes planes}
\label{subsec-planewave}
La transformation de Fourier est très souvent lié avec la périodicité du système étudié.
Il est alors naturel de choisir, dans le contexte de l'approximation numérique,
la base d'ondes planes pour approximer les fonctions d'onde des états propres $\phi_i = \phi_{n,k_i}$.
La théorème de Bloch nous donne:
\begin{equation}
  \phi_{n,k_i} = \frac{1}{\Omega_{maille}}e^{ik_i\cdot r}\sum_G c_{n,k_i}(G)e_{iG\cdot r} = e^{ik_i\cdot r} u_{n,k_i}(r)
\end{equation}
où $u_{n,k_i}(r)$ a la même périodicité que la structure cristalline
et $G$ est un vecteur de réseau dans l'espace réciproque.
La base d'ondes planes nous donne les avantages suivants:
\begin{itemize}
  \item[-] La simplification des calculs de dérivé et d'intégral,
    qui apparaissent dans le calcul des éléments de l'hamiltonien.
  \item[-] Elle nous permet la méthode FFT pour des transformations rapides entre l'espace réel et l'espace réciproque.
  \item[-] L'ensemble d'ondes planes est complèt et orthonormal.
  \item[-] La base est de dimension infinie, mais le coupage de la dimension peut être réalisé facilement
    par la définition d'un seuil d'énergie $E_{cut}$:
    \begin{equation}
      \frac{1}{2}|k+G|^2 \leq E_{cut}
    \end{equation}
    qui relie le nombre d'ondes planes dans la base et $E_{cut}$ par
    \begin{equation}
      N_{op}\propto \Omega_{maille}{(E_{cut})}^{\frac{3}{2}}.
    \end{equation}
    Une étude de la convergence de l'énergie totale du système par rapport à ce seuil de coupage est alors nécessaire.
\end{itemize}

\subsection{Pseudo-potentiel}
\label{subsec-pseudo}
Le nombre total d'électrons du système étudié étant souvent énorme,
beaucoup d'entre eux se trouvent proche de l'atome dans la bonde de valence.
Ensemble avec les atomes, ces électrons dits ``gelé'' contribuent peu àux caractéristiques chimiques du système.
Cependant, ils apparaissent dans les termes d'interaction entre électrons avec les électrons de conduction.
I'idée essentielle du pseudo-potentiel est donc de simuler l'effet de l'atome et des électrons gelés
sur les électrons de conduction par un potentiel effectif qui donne aux électrons de conduction
les même values propres et caractéristiques de ``scattering'' que le système réel.
Un premier avantage de l'introduction de pseudo-potentiel est la réduction du nombre d'électrons dans l'hamiltonien.
Un autre avantage moin évident mais aussi important est la réduction du seuil d'énergie des ondes planes introduit dans~\cref{subsec-planewave},
c'est-à-dire l'augmentation de l'efficacité du calcul, car les pseudo-potentiels sont beaucoup moin abrupts
que les vrais potentiels d'atome qui diverge près de l'origine,
ce qui signifie un moindre nombre de modes nécessaire dans la transformation de Fourier.
Dans notre études, nous utilisons des pour tous les atomes les pseudo-potentiels fournis par le site d'Abinit. %TODO:Give explicite source.

\section{l'Annexe}
Définition des points $k$ dans l'espace réciproque
\begin{itemize}[leftmargin=0em, font=\bfseries]
  \item[ngkpt] Vecteur de taille 3 qui représente le nombre de points de grille dans les trois directions de translation dans l'espace réciproque.
  \item[nshiftk] Nombre de directions dans lesquelle un décalage de point $k$ sera effectué.
  \item[shiftk] Matrice de nshiftk*3 qui précise les directions et les quantité de décalage.
\end{itemize}
Définition de la maille élémentaire:
\begin{itemize}[leftmargin=0em, font=\bfseries]
  \item[acell] Vecteur de taille 3 élément qui représente l'échelle de la taille de la maille élémentaire.
  \item[rprim] Matrice de 3*3 qui représente les trois directions de translation.
    Ensemble avec acell définit les trois vectors de translation dans l'espace réel.
  \item[ntypat] Nombre de types d'atomes dans la maille élémentaire.
  \item[znucl] Vecteur de taille 3 qui représente le charge nucléaire de chaque type d'atome.
  \item[natom] Nombre d'atomes dans la maille élémentaire.
  \item[typat] Vecteur de taille natom qui précise le type de chaque atome dans la maille élémentaire.
  \item[xred] Matrice de natom*3 qui représente la coordonnée réduite de chaque atome,
    par ordre cohérent avec typat.
\end{itemize}
Paramètres de convergence:
\begin{itemize}[leftmargin=0em, font=\bfseries]
  \item[ecut] Seuil d'énergie qui correspont à l'énergie maximale des ondes planes utilisées dans le calcul à un point k donné.
    Plus grande la ecut, plus précis et moin efficace le calcul.
    Une étude sur la convergence en ecut est alors nécessaire pour avoir à la fois la précision et l'efficacité.
  \item[nstep] Nombre maximal d'itération pour le calcul auto-consistente.
    Ce paramètre assure la cessation du calcul, mais doit être suffisamment grande pour que le calcul converge avant de se terminer.
  \item[tolfe] Seuil de la valeur absolue de la différence d'énergie totale entre deux itérations successives.
    On dit que le calcul converge lorsque l'on passe au-dessous de ce seuil.
\end{itemize}
