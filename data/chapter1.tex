\chapter{Introduction et motivation}
Le calcium oxalate est un matériau riche dans la nature. Notamment, on peut le retrouver dans les calculs rénaux (communément appelés ``pierres aux reins'') qui sont formés par le calcium phosphalte, une plus grande structure qui contient les états hydratés du calcium oxalate.
Pour étudier un tel matériau sans changer sa structure ou le détruire, on fait souvent des études spectroscopiques.
Cela consiste à étudier les propriétés optiques et diélectriques du matériau en analysant la réponse du matériau à une lumière incidente.
Typiquement, on peut déduire des spectroscopies de perte et d'absorption la fonction diélectrique (ou plutôt son inverse).
\\\\Dans le cadre du projet en laboratoire de l'École Polytechnique, nous avons choisi de faire le sujet proposé par M. Franceso Sottile du Laboratoire des Solides Iradiés de l'École Polytechnique (LSI), qui consiste à montrer des propriétés spectroscopiques du calcium oxalate par des calculs \textit{ab initio}.
Il s'agit donc de bien comprendre la théorie de la fonctionnelle de la densité (\textit{Density Functional Theory} en anglais, DFT) et la théorie de la fonctionnelle dépendante du temps (\textit{Time-Depending Density Functional Theory} en anglais, TDDFT) et de les appliquer avec les codes existants.
La DFT et la TDDFT sont deux pilliers des calculs \textit{ab initio} qui donnent d'assez bonnes prédictions théoriques par rapport aux expériences réelles.

L'intérêt de notre travail serait de fournir une prédiction théorique et numérique sur les caractéristiques du calcium oxalate en basse énergie qui pourrait être consultée par des expérimentateurs intéressés par ce matériau.
