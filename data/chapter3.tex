\chapter{Simulations numériques}
\label{chap-simulation}
Nos travaux ont été réalisés en deux temps.
Nous avons commencé par étudier l'état fondamental du calcium oxalate à l'aide du logiciel \textit{Abinit}~\cite{Abinit}.
Ensuite, nous avons utilisé le code \textit{DP}~\cite{DP} pour étudier les états excités
en nous servant des résultats obtenus dans l'étape précédente.

Les fonctionnements des outils, aussi bien que les approximations impliquées,
seront présentés au cours de ce chapitre.

\section{Abinit}
\label{sec-abinit}
Abinit est un logiciel qui implémente numériquement la théorie de la fonctionnelle de la densité.
Il nous permet de faire des calculs \textit{ab initio} pour obtenir la densité électronique,
et ainsi l'énergie fondamentale du système.
Plus précisément, Abinit résout le système d'équations de Kohn et Sham de façon auto-consistente à l'aide d'une base d'ondes planes.
À chaque itération, le point d'énergie minimale est calculé par la méthode du gradient conjugué.
L'approximation du pseudo-potentiel est introduite pour éliminer les électrons de cœur dans le système d'équations ,
ce qui réduit considérablement la complexité du problème.

Avant de démarrer le calcul,
plusieurs paramètres doivent être spécifiés pour la résolution numérique.
Du côté physique, le système étudié étant cristallin,
il faut d'abord définir la maille élémentaire.
Du côté technique, il faut préciser les méthodes utilisées et les valuers critiques qui gouvernent le comportement du calcul.
Dans la suite, nous proposerons d'abord quelques approximations importantes. Après,
nous présenterons les paramètres principaux qui nous servent dans nos études.
Une liste complète des paramètres utilisés se trouve dans l'annexe.

\subsection{Espace réciproque}
\label{subsec-reciprocal}
La partie essentielle du calcul avec Abinit consiste à résoudre le système d'équations de Kohn et Sham,
dont la quantité clée est la densité d'électrons $n(\V{r})$.
Si le système étudié est un matériau cristallin,
on pourra y appliquer les conditions périodiques aux limites.
D'après le théorème de Bloch, les états propres, et ainsi les fonctions d'ondes correspondantes,
sont caractérisés par deux nombres quantiques:
le vecteur d'onde $\vb{k}$ dans la première zone de Brillouin, et l'indice de bande $n$.
La densité de charge s'exprime donc par
\begin{equation}
  \label{eqn-den}
  n(\vb{r}) = \sum_{n, \vb{k}}^\V{occ}\abs{\phi_{n, \vb{k}}(\vb{r})}^2,
\end{equation}
avec la somme sur tous les états occupés.
À la limite des cristaux idéals infinis, la densité de points $\vb{k}$ dans l'espace réciproque (de Fourier) tend vers l'infini,
alors l'\cref{eqn-den} devient
\begin{equation*}
  n(\vb{r}) = \sum_n^\V{occ}\int_{\V{ZB}}\dd[3]{\vb{k}}\abs{\phi_{n, \vb{k}}(\vb{r})}^2
\end{equation*}
où ZB signifie la première zone de Brillouin. Ensuite, à l'approximation numérique,
l'intégrale sur $\vb{k}$ doit être discrétisée en somme sur $N_{\vb{k}}$ points $\vb{k}$ pondérés:
\begin{equation*}
  n(\vb{r}) = \sum_n^\V{occ}\sum_{i=1}^{N_{\vb{k}}}\abs{\phi_{n, \vb{k}}(\vb{r})}^2.
\end{equation*}
Le temps de calcul étant proportionnel à $N_{\vb{k}}$,
il est donc nécessaire de choisir les points $\vb{k}$ afin de reproduire le résultat
de l'intégrale avec la moindre erreur et le moindre nombre de points $\vb{k}$.
Une méthode très utilisée est basée sur les travaux de Monkhorst et Pack~\cite{Monkhorst1976}.
Dans notre cas,
l'idée est de prendre d'abord une grille homogène de points $\vb{k}$ dans la zone de Brillouin
qui respecte les symétries du réseau cristallin et d'augmenter ensuite la densité du grillage (et ainsi le nombre total de points $\vb{k}$)
jusqu'à la convergence de l'énergie totale du système.

\subsection{Base d'ondes planes}
\label{subsec-planewave}
La transformation de Fourier est très souvent liée avec la périodicité du système étudié.
Il est alors naturel de choisir, dans le contexte de l'approximation numérique,
une base d'ondes planes pour approximer les fonctions d'onde des états propres $\phi_i = \phi_{n,\vb{k}_i}$.
Le théorème de Bloch nous donne:
\begin{equation*}
  \phi_{n,\vb{k}_i}
  = \frac{1}{\sqrt{\Omega_{\V{maille}}}}e^{i\vb{k}_i\vdot\vb{r}}\sum_{\vb{G}} c_{n,\vb{k}_i}(\vb{G})e^{i\vb{G}\vdot\vb{r}}
  = e^{i\vb{k}_i\vdot\vb{r}} u_{n,\vb{k}_i}(\vb{r})
\end{equation*}
où $u_{n,\vb{k}_i}(\vb{r})$ a la même périodicité que la structure cristalline
et $\vb{G}$ est un vecteur de réseau dans l'espace réciproque.
L'utilisation d'une telle base d'ondes planes a les avantages suivants:
\begin{itemize}
  \item[-] Simplifier les calculs des dérivées et des intégrales pour obtenir les éléments de l'hamiltonien.
  \item[-] Permettre la méthode FFT pour effectuer des transformations rapides entre l'espace réel et l'espace réciproque.
  \item[-] L'ensemble des ondes planes est complèt et orthonormal.
%TODO so what?
  \item[-] La base est de dimension infinie mais peut être réduite facilement
    en définissant une énergie de seuil $E_{\V{cut}}$:
    \begin{equation}
      \frac{1}{2}|\vb{k}+\vb{G}|^2 \leq E_{\V{cut}}
    \end{equation}
    qui relie le nombre des ondes planes dans la base et $E_{\V{cut}}$ par
    \begin{equation}
      N_{\V{op}}\propto E_{\V{cut}}^{\frac{3}{2}}.
    \end{equation}
    Une étude de convergence de l'énergie totale du système par rapport à ce seuil de coupure est alors nécessaire.
\end{itemize}

\subsection{Pseudo-potentiel}
\label{subsec-pseudo}
Le nombre total d'électrons du système étudié étant souvent énorme,
beaucoup d'entre ces électrons se trouvent proche des ions (appelés électrons de cœur).
%TODO : what does parametrique mean exactly? 
Ensemble avec les ions, ces électrons de cœur contribuent seulement de façon paramétrique
aux caractéristiques chimiques du système.
%TODO :??
Cependant, ils apparaissent dans les termes d'interaction entre électrons avec les électrons de conduction.

L'idée essentielle du pseudo-potentiel est donc de simuler l'effet des ions et des électrons de cœur
sur les électrons de valence par un potentiel effectif qui donne aux électrons de valence
%TODO : which eigen vector?
les mêmes valeurs propres et caractéristiques de ``scattering'' que le système réel.
Le premier avantage de l'introduction de pseudo-potentiel est la réduction du nombre de varibles dans l'hamiltonien.
Un autre avantage moins évident mais aussi important est la baisse de l'énergie de seuil des ondes planes introduite dans la \cref{subsec-planewave},
%TODO : I don't know whether we say abrupt in this case, to be verified
c'est-à-dire l'augmentation de l'efficacité du calcul, car les pseudo-potentiels sont beaucoup moins abrupts
que les vrais potentiels d'atome qui diverge près de l'origine,
ce qui signifie que moins de modes sont nécessaires pour la transformation de Fourier.
Dans nos études, nous avons utilisé les pseudo-potentiels fournis par le site d'Abinit pour tous les
quatre types d'atome (Ca, C, O, H)~\cite{Pseudo}.

\subsection{Potentiel d'échange-corrélation}
\label{subsec-xc}
La dernière approximation importante engagée dans l'implémentation numérique
de la méthode de Kohn et Sham est celle sur le potentiel d'échange-corrélation (cf. \cref{eqn-Vtot}).
Par définition, $V_{\V{xc}}[n]$ contient tous les effets au-delà du formalisme de Hartree.
L'\cref{eqn-Vtot} étant exacte, l'expression explicite de $V_{\V{xc}}[n]$ est toujours inconnue,
d'où provient la nécessité l'approximation.

Il existe beaucoup de méthodes d'approximation,
dont certaines sont présentées dans~\cite{Sottile2003}.
À titre d'exemple, 
nous avions la possibilité d'utiliser l'approximation de la densité locale (\textit{Local Density Approximation}, LDA).
L'idée principale de cette dernière consiste à remplaçer la dépendance fonctionnelle de $E_{\V{xc}}[n]$
de la distribution de la densité par celle sur la valeur locale de la densité.
Autrement dit, on suppose que l'énergie d'échange-corrélation du système à un point $\vb{r}$
est égale à celle d'un système homogène qui a la même valeur de la densité d'électrons au point $\vb{r}$.
Par conséquent, $E_{\V{xc}}[n]$ peut s'écrire sous la forme
\begin{equation}
  \label{eqn-exc-lda}
  E_{\V{xc}}^{\V{LDA}}[n] = \int n(\vb{r})\epsilon_{\V{xc}}(n(\vb{r})) \dd{\vb{r}}.
\end{equation}
L'intéret de cette forme est que $\epsilon_{\V{xc}}$ dans l'\cref{eqn-exc-lda} peut être obtenu avec
une précision suffisante par la méthode de Monte-Carlo quantique~\cite{ceperley1980}.
On attend un calcul exact par cette méthode pour un système homogène,
et ainsi de bonne approximation pour les systèmes dont la variation spatiale de la densité d'électrons est faible.
En fait, la portée de la méthode LDA va bien au delà de cette échelle,
et elle s'applique bien dans le domaine du semi-conducteur.
Une démonstration de cet applicabilité est présentée dans~\cite{Sottile2003}.
Les résultats de nos calculs d'Abinit qui seront présentés dans le chapitre suivant sont obtenus avec une autre approximation, la \textit{Generalized Gradient Approximation}, développée par Perdew et ses collègues~\cite{Perdew1986}.

\section{DP}
DP est un code qui permet de calculer la fonction diélectrique
en fonction de la fréquence et du moment transféré.
Dans une simulation d'absorption,
$\vb{q}=0 \text{ et } \omega = \text{la fréquence du photon absorbé}$.
Alors que dans une simulation de perte d'énergie,
$\vb{q}= \text{le moment transféré et } \omega = \text{la perte d'énergie du projectile}$.
Il implémente la théorie détaillée dans la \cref{sec-TDDFT}.

\subsection{Noyau d'échange-corrélation}
Comme dans la méthode de Kohn et Sham (cf. \cref{subsec-KS}),
un terme appelé le noyau d'échange-corrélation est introduit (cf. \cref{eqn-chi0chi})
pour tenir en compte tous les effets dynamiques d'échange et de corrélation
exercés par le potentiel externe.
À nouveau, ce terme doit être approximé car son expression explicite est inconnue.
Les méthodes d'approximation que nous avons utilisées dans le cadre de ce projet sont
la \textit{Random Phase Approximation} (RPA) et
la \textit{Adiabatic Local Density Approximation} (ALDA).
La première approche, qui est aussi la plus simplifiée, prend un noyau nulle et
néglige donc les effets d'échange et de corrélation mentionnés ci-dessus.
La seconde approche, qui est largement utilisée,
prend comme noyau la dérivée fonctionnelle du potentiel d'échange-corrélation
défini dans la \cref{subsec-KS}:
\begin{equation*}
  f_{\V{xc}}^{\V{ALDA}}(\vb{r}, \vb{r}')
  = \delta(\vb{r}, \vb{r}')\pdv{V_{\V{xc}}^{\V{LDA}}(n(\vb{r}), \vb{r})}{n(\vb{r})}.
\end{equation*}
