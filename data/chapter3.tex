\chapter{Simulations numériques}
\label{chap-simulation}
Nos travaux ont été réalisés en deux temps.
Nous commençons par étudier l'état fondamental du calcium oxalate à l'aide du logiciel \textit{Abinit}~\cite{Abinit}.
Après, nous utilisons le code \textit{DP}~\cite{DP} pour étudier les états excités
en servant des résultats obtenus dans l'étape précédente.

Les fonctionnements des outils, aussi bien que les approximations impliquées,
seront présentées au cours de ce chapitre, suivies par l'explicitation du schéma de la simulation.

\section{Abinit}
\label{sec-abinit}
Abinit est un logiciel qui implémente numériquement la théorie de la fonctionnelle de la densité.
Il nous permet de faire des calculs ab initio pour obtenir la densité électronique,
et ainsi l'énergie fondamentale du système.
Plus précisément, Abinit effectue la résolution auto-consistente pour le système d'équations de Kohn et Sham à l'aide une base d'ondes planes.
A chaque itération, le point minimum d'énergie est calculé par la méthode du gradient conjugué,
et l'approximation du pseudo-potentiel est introduite pour éliminer dans le systéme d'équations les électrons de cœur,
ce qui réduit considérablement la complexité du probleme.

Avant de démarer le calcul,
plusieurs paramètres doivent être spécifiés pour la résolution numérique.
Du côté physique, le système étudié étant cristallin,
il faut d'abord définir la maille élémentaire.
Du côté technique, il faut préciser les méthodes utilisées et des valuers critiques qui gouvernent le comportement du calcul.
Dans la suite, nous proposons d'abord quelques approximations importantes. Ensuite,
Nous présentons les paramètres principaux qui nous servent dans notre études.
Une liste complète des paramètres utilisés se trouve dans l'annexe.

\subsection{Espace réciproque}
\label{subsec-reciprocal}
La partie éssentielle du calcul avec Abinit consiste à résoudre le système d'équations de Kohn et Sham,
dont la quantité clée est la densité d'électrons $n(\V{r})$.
Or les système étudiés sont souvent des matériaux cristallins,
auxquels on applique la condition périodique aux limites.
D'après le théorème de Bloch, les états propres, et ainsi les fonctions d'ondes correspondantes,
sont caractérisés par deux nombres quantiques:
le vecteur d'onde $\vb{k}$ dans la première zone de Brillouin, et l'indice de bande $n$.
La densité de charge s'exprime donc par
\begin{equation}
  \label{eqn-den}
  n(\vb{r}) = \sum_{n, \vb{k}}^\V{occ}\abs{\phi_{n, \vb{k}}(\vb{r})}^2,
\end{equation}
avec la somme sur tous les états occupés.
A la limite des cristaux idéals infinis, la taille de grille dans l'espace réciproque tend vers zéro,
alors l'\cref{eqn-den} devient
\begin{equation*}
  n(\vb{r}) = \sum_n^\V{occ}\int_{\V{ZB}}\dd[3]{\vb{k}}\abs{\phi_{n, \vb{k}}(\vb{r})}^2
\end{equation*}
où ZB signifie la première zone de Brillouin. Ensuite, à l'approximation numérique,
l'intégral sur $\vb{k}$ doit être discrétisée en somme sur $N_{\vb{k}}$ point $\vb{k}$ pondérés:
\begin{equation*}
  n(\vb{r}) = \sum_n^\V{occ}\sum_{i=1}^{N_{\vb{k}}}\abs{\phi_{n, \vb{k}}(\vb{r})}^2.
\end{equation*}
Le temps de calcul étant proportionnel à $N_{\vb{k}}$,
il est donc nécessaire de choisir les points $\vb{k}$ afin de reproduire le résultat
de l'intégral avec la moindre erreur et le moindre nombre de point $\vb{k}$.
Une méthode très utilisée est basée sur les travaux de Monkhorst et Pack~\cite{Monkhorst1976}.
L'idée est d'abord de prendre une grille homogène de points $\vb{k}$ dans la zone de Brillouin
qui respecte les symétries du réseau cristallin,
et ensuite d'augmenter la densité du grillage (et ainsi celle du nombre total de point $\vb{k}$)
jusqu'à la convergence, par example, de l'énergie totale du système.

\subsection{Base d'ondes planes}
\label{subsec-planewave}
La transformation de Fourier est très souvent lié avec la périodicité du système étudié.
Il est alors naturel de choisir, dans le contexte de l'approximation numérique,
la base d'ondes planes pour approximer les fonctions d'onde des états propres $\phi_i = \phi_{n,\vb{k}_i}$.
La théorème de Bloch nous donne:
\begin{equation*}
  \phi_{n,\vb{k}_i}
  = \frac{1}{\sqrt{\Omega_{\V{maille}}}}e^{i\vb{k}_i\vdot\vb{r}}\sum_{\vb{G}} c_{n,\vb{k}_i}(\vb{G})e^{i\vb{G}\vdot\vb{r}}
  = e^{i\vb{k}_i\vdot\vb{r}} u_{n,\vb{k}_i}(\vb{r})
\end{equation*}
où $u_{n,\vb{k}_i}(\vb{r})$ a la même périodicité que la structure cristalline
et $\vb{G}$ est un vecteur de réseau dans l'espace réciproque.
La base d'ondes planes nous donne les avantages suivants:
\begin{itemize}
  \item[-] La simplification des calculs de dérivé et d'intégral,
    qui apparaissent dans le calcul des éléments de l'hamiltonien.
  \item[-] Elle nous permet la méthode FFT pour des transformations rapides entre l'espace réel et l'espace réciproque.
  \item[-] L'ensemble d'ondes planes est complèt et orthonormal.
  \item[-] La base est de dimension infinie, mais le coupage de la dimension peut être réalisé facilement
    par la définition d'un seuil d'énergie $E_{\V{cut}}$:
    \begin{equation}
      \frac{1}{2}|\vb{k}+G|^2 \leq E_{\V{cut}}
    \end{equation}
    qui relie le nombre d'ondes planes dans la base et $E_{\V{cut}}$ par
    \begin{equation}
      N_{\V{op}}\propto E_{\V{cut}}^{\frac{3}{2}}.
    \end{equation}
    Une étude de la convergence de l'énergie totale du système par rapport à ce seuil de coupage est alors nécessaire.
\end{itemize}

\subsection{Pseudo-potentiel}
\label{subsec-pseudo}
Le nombre total d'électrons du système étudié étant souvent énorme,
beaucoup d'entre eux se trouvent proche des ions.
Ensemble avec les ions, ces électrons de cœur contribuent seulement de façon paramétrique
àux caractéristiques chimiques du système.
Cependant, ils apparaissent dans les termes d'interaction entre électrons avec les électrons de conduction.
I'idée essentielle du pseudo-potentiel est donc de simuler l'effet des ions et des électrons de cœur
sur les électrons de valence par un potentiel effectif qui donne aux électrons de valence
les même values propres et caractéristiques de ``scattering'' que le système réel.
Un premier avantage de l'introduction de pseudo-potentiel est la réduction du nombre d'électrons dans l'hamiltonien.
Un autre avantage moin évident mais aussi important est la réduction du seuil d'énergie des ondes planes introduit dans la \cref{subsec-planewave},
c'est-à-dire l'augmentation de l'efficacité du calcul, car les pseudo-potentiels sont beaucoup moin abrupts
que les vrais potentiels d'atome qui diverge près de l'origine,
ce qui signifie un moindre nombre de modes nécessaire dans la transformation de Fourier.
Dans notre études, nous utilisons les pseudo-potentiels fournis par le site d'Abinit pour toute les
quatre espèces d'atome (Ca, C, O, H)~\cite{Pseudo}.

\subsection{Potentiel d'échange-corrélation}
\label{subsec-xc}
Une dernière approximation importante engagé dans l'implémentation numérique
de la méthode de Kohn et Sham est celle sur l'énergie d'échange-corrélation (\cref{eqn-exc}).
Par définition, $E_{\V{xc}}[n]$ contient tous les éffets au delà du formalisme de Hartree.
L'\cref{eqn-exc} étant exacte, l'expression explicite de $E_{\V{xc}}[n]$ est toujours inconnue,
d'où provient l'approximation.

Il existe beaucoup de méthodes d'approximation,
dont certaines sont présentées dans~\cite{Sottile2003}.
Dans nos études, l'approximation utilisée est celle de la densité locale (\textit{Local Density Approximation}, LDA),
l'idée principale de laquelle consiste à remplaçer la dépendance fonctionnelle de $E_{\V{xc}}[n]$
sur la distribution de la densité par celle sur la valeur locale de la densité.
Autrement dit, on suppose que l'énergie d'échange-corrélation du système à un point $\vb{r}$
est égale à celle d'un système homogène qui a la même valeur de la densité d'électrons au point $\vb{r}$.
Par conséquent, $E_{\V{xc}}[n]$ peut s'écrire sous la forme
\begin{equation}
  \label{eqn-exc-lda}
  E_{\V{xc}}^{\V{LDA}}[n] = \int n(\vb{r})\epsilon_{\V{xc}}(n(\vb{r})) \dd{\vb{r}}.
\end{equation}
L'intéret de cette forme est que $\epsilon_{\V{xc}}$ dans l'\cref{eqn-exc-lda} peut être obtenu avec
une précision suffisante par la méthode de Monte-Carlo quantique~\cite{ceperley1980}.
On attend un calcul exact par cette méthode pour un système homogène,
et ainsi de bonne approximation pour les systèmes dont la variation spatiale de la densité d'électrons est faible.
En fait, la portée de la méthode LDA va bien au delà de cette échelle,
et elle s'applique bien dans le domaine du semi-conducteur.
Une démonstration de cet applicabilité est présentée dans~\cite{Sottile2003}.

\section{DP}
DP est un code qui permet de calculer le coefficient diélectrique
en fonction de la fréquence et du moment transféré.
Dans le cas d'une simulation d'absorption,
$\vb{q}=0 \text{et} \omega = \text{la fréquence du photon absorbé}$.
Alors que dans le cas d'une simulation de perte d'énergie,
$\vb{q}= \text{le moment transféré et} \omega = \text{la perte d'énergie du projectile}$.
Il implémente la théorie détaillée dans la \cref{sec-TDDFT}.

\subsection{Noyau d'échange-corrélation}
Comme dans la méthode de Kohn et Sham (\cref{subsec-KS}),
un terme dit le noyau d'échange-corrélation est introduit (\cref{eqn-chi0chi})
pour tenir en compte tous les effets dynamiques d'échange et de corrélation
exercés par le potentiel externe.
À nouveau, ce terme doit être approximé car son expression explicite est inconnue.
Les méthodes d'approximation que nous utilisons dans le cadre de ce projet est
la \textit{Random Phase Approximation} (RPA) et
la \textit{Adiabatic Local Density Approximation} (ALDA).
La première approche, qui est aussi la plus simplifiée, prend un noyau nulle et
néglige donc les effets mentionnés ci-dessus.
La seconde approche, qui est largement utilisée,
prend le noyau comme la dérivée fonctionnelle du potentiel d'échange-corrélation
définie dans la \cref{subsec-KS}:
\begin{equation*}
  f_{\V{xc}}^{\V{ALDA}}(\vb{r}, \vb{r}')
  = \delta(\vb{r}, \vb{r}')\pdv{V_{\V{xc}}^{\V{LDA}}(n(\vb{r}), \vb{r})}{n(\vb{r})}.
\end{equation*}
