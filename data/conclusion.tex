\chapter{Conclusion, perspective et divers}
Dans ce projet, nous avons trouvé, par des calculs \textit{ab initio}, les caractéristiques spectroscopiques de l'oxalate de calcium. 
Ce travail pourrait fournir une référence théorique aux expérimentateurs intéressés à ce matériau.
La continuation de notre travail serait de traiter les oxalates de calcium hydratés (avec une, deux ou trois molécules d'eau dans la cellule élémentaire) qui ont des structures moléculaires plus complexes et des cellules élémentaires plus grandes par manque de symétries spatiales de ces structures.
Nous avons eu aussi la possibilité de commencer cette partie en lançant les calculs pour l'oxalate de calcium monohydraté et l'oxalate de calcium trihydraté. 
Cette suite serait prise par notre tuteur du projet, M. Francesco Sottile, ou par d'autres élèves intéressés. 
\\\\Tout au long de ce projet, nous avons eu un premier aperçu sur la méthode de la DFT et celle de la TDDFT qui font partie des outils puissants et couramment utilisés pour résoudre les problèmes à plusieurs corps. 
D'autre part, nous avons appris à manipuler des codes de calcul scientifique.
Enfin, nous avons compris, grâce à ce projet de recherche en laboratoire, comment travailler en binôme de façon efficace.
L'objectif pédagogique de ce projet est bien atteint à notre point de vue.