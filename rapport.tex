\documentclass[12pt]{report}

\usepackage{amsmath,amsthm,verbatim,amssymb,amsfonts,amscd, graphicx}
\usepackage[utf8]{inputenc}
\usepackage{enumitem}
\usepackage{amssymb}
\usepackage{amsmath}
\usepackage{mathrsfs}
\usepackage{graphicx}
\usepackage{yfonts}
\usepackage{amscd}
\graphicspath{{images/}}
%%%%%%%%%%%%%%%%%%%%%%%%%%%%%%%%%%%%%%%%%%%%%%%%%%%%%%%%%%%%%%%%%%%%%%%%%%%%%%
\renewcommand{\contentsname}{Table des matières}
%\renewcommand{\refname}{Bibliographie}
\renewcommand{\baselinestretch}{1.05}

\theoremstyle{theoreme}
\newtheorem{theoreme}{Théorème}

%%%%%%%%%%%%%%%%%%%%%%%%%%%%%%%%%%%%%%%%%%%%%%%%%%%%%%%%%%%%%%%%%%%%%%%%%%%%%
\title{PRL}

\author{En-Hung Chao \\ Honghao Li}
\date{\today}
%%%%%%%%%%%%%%%%%%%%%%%%%%%%%%%%%%%%%%%%%%%%%%%%%%%%%%%%%%%%%%%%%%%%%%%%%%%%%%
%%%%%%%%%%%%%%%%%%%%%%%%%%%%%%%%%%%%%%%%%%%%%%%%%%%%%%%%%%%%%%%%%%%%%%%%%%%%%%
\begin{document}

\maketitle

\newpage
\tableofcontents

\newpage
\chapter{Introduction}
\addcontentsline{toc}{section}{Introduction}


%%%%%%%%%%%%%%%%%%%%%%%%%%%%%%%%%%%%%%%%%%%%%%%%%%%%%%%%%%%%%%%%%%%%%%%%
\chapter{Études théoriques des systèmes à plusieurs corps
\section{Théorie de la fonctionelle de la densité}
\subsection{Approximation d'Oppenheimer}
Dans la physique des solides, on est ramené à étudier l'équation de Schrödinger d'un système à plusieurs corps ($N$ électrons et $M$ ions par exemple), ce qui consiste à résoudre une équation du type

$$
(\sum_{i=1}^N \frac{\textbf{p}_i^2}{2m_i} + \sum_{I=1}^M \frac{\textbf{P}_i^2}{2M_I} 
+ \sum_{i<j}\frac{e^2}{| \textbf{r}_i - \textbf{r}_j |} - \sum_{i, I}\frac{Z_I e^2}{| \textbf{r}_i 
- \textbf{R}_I |} + \sum_{I<J}\frac{Z_I Z_J e^2}{| \textbf{R}_i - \textbf{R}_j |} ) \Psi(\textbf{r}, \textbf{R}) = E_{tot} \Psi(\textbf{r}, \textbf{R})
$$
où les lettres en majuscule sont des quantités physiques pour les ions, les lettres en miniscule sont celles pour les électrons et $E_{tot}$ une certaine énergie propre du système qu'on voudra trouver.

Comme le système qu'on considère (les solides) comprend souvent un nombre très élevé d'électrons et d'ions, cette équation devient très difficile à résoudre numériquement. Ainsi, on doit appliquer certaines approximations au système afin de réduire le temps de calcul. La première approximation qu'on peut prendre, celle la plus intuitive, est celle de Born-Oppenheimer. En tenant compte du rapport de masse entre les électrons et les ions et en supposant que les ions sont quasiment figés par rapport aux électrons (ayant une longueur d'onde de De Broglie plus élevée), on peut réécrire l'hamiltonien du système comme

$$
H = \underbrace{\sum_{i=1}^N \frac{\textbf{p}_i^2}{2m_i}}_{T}
+ \underbrace{\sum_{i<j}\frac{e^2}{| \textbf{r}_i - \textbf{r}_j |}}_{V}
- \underbrace{\sum_{i, I}\frac{Z_I e^2}{| \textbf{r}_i - \textbf{R}_I |} + \sum_{I<J}\frac{Z_I Z_J e^2}{| \textbf{R}_i - \textbf{R}_j |}}_{V_{ext}}
$$ 
où $T$ est l'énergie cinétique des électrons, $V$ leur énergie potentielle d'interaction et $V_{ext}$ l'énergie potentielle dûe à l'extérieur. 
Notre problème devient 

$$
H \Psi_e(\textbf{r}_1, \ldots \textbf{r}_N) = E_{tot} \Psi_e(\textbf{r}_1, \ldots \textbf{r}_N)
$$

Or, le problème n'est toujours pas simple car la fonction d'onde $\Psi_e$ que l'on cherche est une fonction à plusieurs variables. La méthode proposée par Kohn et Sham~\cite{Koh65} que l'on va présenter dans la suite nous contournent alors le problème.

\subsection{La méthode de Kohn et Sham}
Si les électrons étaient non-interagissant entre eux, on pourrait écrire l'hamiltonien comme une somme de plusieurs hamiltoniens d'un seul électron, \textit{i.e.},

$$
H = \sum_i H_i^{(1e)} = \sum_i (-\frac{\hbar^2}{2m}\nabla_i^2 + V_{ext}(\textbf{r}_i) + \frac{1}{2}\sum_{i \neq j} v_{ij}(|\textbf{r}_i - \textbf{r}_j))
$$

Le théorème développé par Hohenberg et Kohn~\cite{Hoh64} nous donne la possibilité d'approximer la vraie solution en passant par un système auxiliaire d'électrons non-interagissants.

\begin{theoreme}[Hohnenberg-Kohn]
La valeur d'espérance d'un observable dans l'état fondamental est l'unique fonctionnel de la densité d'électron. 
\end{theoreme}

La preuve de ce théorème peut être trouvée dans~\cite{Sot03}. 
Ce résultat implique que la valeur d'espérance de l'hamiltonien à l'état fondament total, donc l'énergie fondamentale du système, est une fonctionnel de la densité d'électrons $n(\textbf{r}) = |\Psi_e(\textbf{r}_1, \ldots \textbf{r}_N) |^2$.

La méthode de Kohn et Sham consiste à poser un système auxiliaire d'hamiltonien $H' = T' + V'_{ext}$, qui a le même densité d'électrons que notre vrai système $H = T + V + V_{ext}$. Ce système auxiliaire étant non-interagissant, on peut donc prendre pour la densité $n(\textbf{r}) = n'(\textbf{r}) = \sum_i |\phi_i(\textbf{r})|^2$. On peut donc considérer le système d'équation suivant

$$
(-\frac{1}{2}\nabla_i^2 + V_{tot}[n](\textbf{r}))\phi_i(\textbf{r}) = \epsilon_i \phi_i (\textbf{r})
$$

avec 
$$
V_{tot}(\textbf{r}) = V_{ext}(\textbf{r}) + \int d\textbf{r}' \frac{n(\textbf{r}')}{|\textbf{r} - \textbf{r'}|} + V_{xc}([n], \textbf{r})
$$
pour chacun des électrons $i$. Le problème devient un simple problème de diagonalisation. On pourra ensuite faire une résolution auto-consistente en réinjectant $n = \sum_i |\phi_i^{(p)}(\textbf{r})|^2 $ après chaque itération $p$. (L'auto-consistence de cette méthode n'a toujours pas été prouvée.) Une fois minisé l'énergie totale du système auxiliaire $\langle \Psi_e | H' | \Psi_e \rangle $, on obtiendra la densité exacte $n(\textbf{r})$ du vrai système, ce qui nous permettra de trouver l'énergie de l'état fondamental.

Cependant, il y a une approximation à faire pour le terme du potentiel d'échange et de corrélation $V_{xc}(\textbf{r}) = \frac{\delta E_{xc}}{\delta n(\textbf{r})}$, où 

$$
E_{xc} = T[n] + V[n] - \frac{1}{2}\int d\textbf{r} \int d\textbf{r}' v(\textbf{r}, \textbf{r}') n(\textbf{r}) n(\textbf{r'}) - T'[n]
$$

Il existe beaucoup de méthodes d'approximation, par exemple celle de LDA (Local Density Approximation). Certaines méthodes d'approximations sont présentées dans \cite{Sot03} (Sect. 2.4 et 2.5).

%17/02/2017%%%%%%%%%%%%%%%%%%%%%%%%%%%%%%%%%%%%%%%%%%%%%%%
\section{Théorie de la fonctionelle de la densité dépendente du temps} \label{TDDFT}
Dans les études de spectroscopie, nous étudions les comportements des matériaux sous exicitation. La constante diélectrique et son inverse sont par exemple des grandeurs physiques que l'on regarde souvent dans les expériences (déduites à partir de l'absorption et de la réflexion des lumières, cf. \cite{Sot03} Chap. 1). Par conséquent, la simple étude de l'état fondamental ne suffit pas, il nous faut de plus des études sur les états excités. 

La méthode que nous allons présenter dans ce rapport, la théorie de la fonction de densité dépendente du temps (\textit{Time-Depending Density Functional Theory}, est utilisée dans le code à l'aide duquel on fait des simulations numériques \textit{ab initio}. Pour l'introduire, nous commençons par présenter les éléments de la théorie de la réponse linéaire.

\subsection{Théorie de la réponse linéaire}
Soit $n$ un observable. On suppose que $n$ varie en fonction de la perturbation $V$ de par 

$$
n(\textbf{r}, t) = \int dt d\textbf{r'}  \chi (\textbf{r}, \textbf{r'}, t-t') V(\textbf{r'}, t')
$$

La fonction $\chi$ s'appelle la \textit{foncition de réponse linéaire}. Dans le cas où le hamiltonien de perturbation dépend aussi de $n$ par la relation suivante

$$
H = \int n(\textbf{r}, t) V(\textbf{r}, t) d\textbf{r}
$$

la fonction de réponse linéaire $\chi$ est donnée par la formule de Kubo\cite{Ton12}

\begin{equation}\label{kubo}
\chi (\textbf{r}, \textbf{r'}, t-t') = -i \langle N | [n(\textbf{r}, t), n(\textbf{r'}, t) |N \rangle \Theta(t-t')
\end{equation}

où $|N \rangle $ désigne l'état dans lequel on mesure l'observable $n$.

Nous supposons que la structure de bande obtenue à partir de l'état fondamental nous fournisse une base complète. Nous partons de $|N \rangle$ l'état fondamental constitué de $N$ électrons dans \ref{kubo}. Notons $| j_N \rangle $ le j-ième état excité de notre système à $N$ électrons. En faisant la transformation de Fourier temporelle, la fonction de réponse linéaire s'écrit dans ce cas 

\begin{equation}
\chi (\textbf{r}, \textbf{r}', \omega) = \sum_j \frac{f_j(\textbf{r}) f_j^*(\textbf{r}') }{\omega - \Omega_j + i \eta} + \frac{f_j(\textbf{r}') f_j^*(\textbf{r}) }{\omega + \Omega_j + i \eta}
\end{equation}

où $f_j = \langle N | n(\textbf{r}, 0) | j_N \rangle$ le nombre d'occupation de l'état $j$, $\Omega_j \pm (E_0 - E_j)$ le \textit{gap} entre l'état $j$ qui et l'état fondamental et enfin $\eta$ le terme de lissage (\textit{broadening}) qui est à tendre vers 0 pour le calcul théorique.

Notons $n$ la fonciton de densité. C'est un observable obtenu à partir de la relation suivante 

$$
n = \psi^\dagger \psi
$$

où $\psi$ est l'opérateur d'annihilation et $\psi^\dagger$ l'opérateur de création de l'état que l'on considère (en analogie avec les oscillateurs harmoniques).

On peut développer $\psi$ (et donc $\psi^\dagger$) dans la base des opérateurs de création
pour un seul électron.

$$
\psi = \sum_\textbf{k} \phi_\textbf{k}(\textbf{r}) a_\textbf{k}
$$

où $\phi_\textbf{k}$ est la fonction d'onde de Bloch pour l'impulsion $\textbf{k}$ et $a_\textbf{k}$ l'opérateur de création correspondant. La fonction de réponse linéaire peut encore s'écrire

\begin{equation}\label{chi}
\chi(\textbf{r}, \textbf{r}', \omega) = \sum_{vc} \frac{(f_v - f_c)\phi_v(\textbf{r}) \phi_c^*(\textbf{r}) \phi_v^*(\textbf{r}') \phi_c(\textbf{r}')}{\omega - (\epsilon_c - \epsilon_v) + i\eta} \quad + \quad a.r.
\end{equation}

où $a.r.$ représente les termes d'anti-résonance et $v$ et $c$ pour bande de valence et bande de conduction respectivement.

\subsection{TDDFT}
Nous allons étudier le comportement du coefficient diélectrique du système en présence d'une excitation extérieure. La relation entre le potentiel d'perturbation extérieure et le potentiel total à l'intérieur du système est donnée par (dans l'espace réciproque de transformation de Fourier temporelle)

$$
V_{tot}(\textbf{r}, \omega) = \int d\textbf{r}' \epsilon^{-1}(\omega) V_{ext}(\textbf{r}', \omega)
$$  

On notera $\chi^0$ et $\chi$ les fonctions de réponse linéaire pour $V_{tot}$ et $V_{ext}$ respectivement. On a les relations entre la fonctionelle de densité $n$ et ces deux potentiels

$$
n(\textbf{r}, t) = \int d\textbf{r}' dt' \chi^0(\textbf{r}, \textbf{r'}, t-t') V_{tot}(\textbf{r}', t) = \int d\textbf{r}' dt' \chi(\textbf{r}, \textbf{r'}, t-t') V_{ext}(\textbf{r}', t)
$$

\subsection{Coefficient diélectrique et fonction de perte}

%%%%%%%%%%%%%%%%%%%%%%%%%%%%%%%%%%%%%%%%%%%%%%%%%%%%%%%%%%
\chapter{Simulations numériques}
Nos travaux ont été réalisés en deux temps. Nous commençons par étudier l'état fondamental du calcium oxalate. Le code utilisé s'appelle \textit{Abinit}. Ensuite, nous utilisons le fichier décrivant la fonction de densité de l'état fondamental obtenu dans la première étape pour calculer le coefficient diélectrique dont on déduit la fonction de perte (\textit{Loss fonction} en anglais). Le code utilisé pour cette étape s'appelle \textit{DP}. 
%TODO une présentation d'Abinit et de DP 

Les fonctionnements des outils seront présentés au cours de ce chapitre.

\section{Abinit}

\section{DP}  
%DP est un code qui permet de calculer le coefficient diélectrique en fonction de la fréquence de lumière absorbée par le matériau. Il implémente le calcul détaillé dans la section \ref{TDDFT}. Nous donnons une brève description de son fonctionnement par la suite.

\section{Simulation}

Pour toute étude $\textit{ab initio }$des matériaux, il faut trouver un optimum entre le temps de calcul et la précision de la convergeance. Il nous faut donc d'abord les différents échantillages de notre fonction de densité dans la première zone de Brouillon. Pour le cas du calcium oxalate, nous avons généré trois fichiers avec Abinit, qui contiennent respectivement 16 (avec symétries), 120 (avec symétries) et 480 (grillage translaté et sans symétries) points-k.





\appendix
\chapter{Paramètres d'entrée de DP}


\begin{itemize}
\item[rda, alda] La méthode d'approximation pour le terme du potentiel d'échange et corrélation, \textit{i.e.}, le terme $f_{xc}$ %TODO referer à l'expression de chi0
\item[wfnsh] Le nombre de fonctions dont on a besoin pour faire la transformation de Fourier (wfnsh)%TODO ref to eq
\item[nbands] Le nombre de bandes pour lesquelles les transitions sont autorisées (nbands)
\item[matsh] Le nombre de fonctions dont on a besoin pour faire l'inversion de la matrice afin d'obtenir le coefficient $\chi_0$
\item[lomo] La bande la plus basse à partir de laquelle on autorise les transitions
\item[q] La direction des transitions
\item[omegai] La limite basse de la plage d'énergie considérée
\item[omegae] La limite haute de la plage d'énergie considérée
\item[domega] Le pas de l'échantillonage de la plage d'énergie
\item[broad] Le paramètre de lissage, qui correspond à $\eta$ dans l'expression de $\chi$. Ce paramètre est lié au temps de vie des états considérés. En effet, dans la théorie, on fait tendre ce terme-là vers 0, ce qui signifie que les états que l'on considère peuvent exister en permanant. Or, en pratique, il est très rare d'avoir des états permanants à cause des perturbations. On adjuste donc ce terme par rapport aux résultats expérimentaux. Plus ce terme est grand, plus le spectre va être lisse. L'importance de ce paramètre est surtout de donner des résultats qualitativement corrects.
\item[novkb] Paramètre à mettre si le calcul du terme commutateur est omis. Plus précisément, %TODO
\item[shiftk] Paramètre à mettre dans le cas où aucune symétrie n'est utilisée pour générer la grille des points-k dans le fichier décrivant la fonction de densité de l'état fondamental.
\end{itemize}


%%%%%%%%%%%%%%%%%%%%%%%%%%%%%%%%%%%%%%%%%%%%%%%%%%%%%%%%%%%%%%%%%%

\begin{thebibliography}{9}

\bibitem{Sot03}
	F. Sottile, thèse de doctorat (2003)

\bibitem{Hoh64}
	P. Hohenberg et W. Kohn, Phys. Rev. Lett. 136, B864 (1964)

\bibitem{Koh65}
	W. Kohn et L. J. Sham, Phys. Rev. Lett. 140, A1133 (1965)
	
\bibitem{Ton12}
	D. Tong, Kinetic Theory, cours donné dans University of Cambridge Graduate Course, 2012

\end{thebibliography}


\end{document}